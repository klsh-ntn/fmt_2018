%!TEX TS-program = xelatex
%!TEX encoding = UTF-8 Unicode

\documentclass[a4paper, 12pt]{article}

%\newcommand{\LastChange}{Time-stamp: "2014-07-09 15:59:43 aga SeminarProblems.tex"}

\usepackage[includehead, top=0.2cm, bottom=0.2cm, left=1cm, right=1cm]{geometry}

\usepackage[colorinlistoftodos]{todonotes}
\usepackage[colorlinks=true, allcolors=blue]{hyperref}

%\usepackage{euler}
%\usepackage{xltxtra} % loads: fixltx2e, metalogo, xunicode, fontspec

\usepackage{multicol} % many columns
\usepackage{amsmath,amsfonts,amssymb,amsthm}
\usepackage{fullpage}
\usepackage{graphicx}
\usepackage{bm}

%\usepackage{enumerate}
\usepackage{textcomp} % text in formulas

%\usepackage{paralist}
\usepackage{enumitem} % more options for lists

\usepackage{tikz} % картинки
\usetikzlibrary{arrows.meta} % tikz-прибамбас для рисовки стрелочек подлиннее


\usepackage{fontspec} % что-то про шрифты?
\usepackage{polyglossia} % русификация xelatex

\setmainlanguage{russian}
\setotherlanguages{english}

% download "Linux Libertine" fonts:
% http://www.linuxlibertine.org/index.php?id=91&L=1
\setmainfont{Linux Libertine O} % or Helvetica, Arial, Cambria
% why do we need \newfontfamily:
% http://tex.stackexchange.com/questions/91507/
\newfontfamily{\cyrillicfonttt}{Linux Libertine O}

\defaultfontfeatures{Mapping=tex-text}

\AddEnumerateCounter{\asbuk}{\russian@alph}{щ} % для списков с русскими буквами
\setlist[enumerate, 2]{label=\asbuk*),ref=\asbuk*}

%\setmainfont{Times New Roman}
%\setmainfont{Arial}
%\setmainfont{PT Sans}


%\setlength{\topmargin}{0in}
%\setlength{\headheight}{0cm}
%\setlength{\headsep}{0in}
%\setlength{\oddsidemargin}{-0.5in}
%\setlength{\evensidemargin}{-0.5in}
%\setlength{\textwidth}{7.5in}
%\setlength{\textheight}{9.0in}


% \newcommand{\staritem}{\refstepcounter{enumi}\item[\bf *\theenumi.]}

% \newcommand{\bsym}{\boldsymbol}


%\newcommand{\FigWidth}{0.3\columnwidth}



\newtheoremstyle{break}% name
  {}%         Space above, empty = `usual value'
  {14pt}%         Space below
  {}% Body font
  {}%         Indent amount (empty = no indent, \parindent = para indent)
  {\bfseries}% Thm head font
  {.}%        Punctuation after thm head
  {\newline}% Space after thm head: \newline = linebreak
  {}%         Thm head spec

\theoremstyle{break}
\newtheorem{problem}{Задача}[subsection]
\renewcommand{\theproblem}{\arabic{problem}}% Remove subsection from the counter representation


\begin{document}

\thispagestyle{empty}
%%%%%%%%%%%%%%%%%%%%%%%%%%%%%%
%%%%%%%%%%%%%%%%%%%%%%%%%%%%%%
\subsection*{Первый Тур}
%%%%%%%%%%%%%%%%%%%%%%%%%%%%%%


\begin{problem}
%SchoolNova
Реши уравнение:
\[
2018 − 2(2018 − 2(2018 − 2x)) = x
\]
\end{problem}

\begin{problem}
Три бруска массами $m$, $3m$ и $2m$, соответственно, толкают с силой $F$.
Бруски касаются друг друга и скользят по плоскости без трения.
Найди величину силы, с которой второй брусок действует на первый.

\begin{minipage}{0.8\textwidth}
\begin{center}
\resizebox{8.0cm}{1.6cm}{
\begin{tikzpicture}
\draw (-5,0) -- (5,0);
\draw[-{Latex[length=5mm, width=2mm]}] (-5,1) -- (-3,1) node[midway, above]{$F$};
\draw (-3,0) -- (-3,2) -- (-1,2) node[midway, below=0.8cm]{$m$} node[midway, above]{1} -- (-1,0);
\draw (-1,0) -- (-1,2) -- (1,2) node[midway, below=0.8cm]{$3m$} node[midway, above]{2} -- (1,0);
\draw (1,0) -- (1,2) -- (3,2) node[midway, below=0.8cm]{$2m$} node[midway, above]{3} -- (3,0);
\end{tikzpicture}
}
\end{center}
\end{minipage}
\end{problem}


\begin{problem}
% задачка от Ламзина
В треугольнике $ABC$ угол между биссектрисами, исходящими из углов $A$ и $B$,
равен $43^{\circ}$. Найди угол $C$ треугольника $ABC$.
\end{problem}


\begin{problem}
Три одинаковых шарика надувают гелием так, что их диаметры относятся как 1:2:3.
Известно, что первый шарик может поднять груз максимальной массой $m$,
а второй — максимальной массой $15m$. Считать плотность гелия во всех шариках одинаковой.

Какой максимальный груз может поднять третий шарик?
\end{problem}

\subsection*{Первый Тур}
%%%%%%%%%%%%%%%%%%%%%%%%%%%%%%
\setcounter{problem}{0}

\begin{problem}
%SchoolNova
Реши уравнение:
\[
2018 − 2(2018 − 2(2018 − 2x)) = x
\]
\end{problem}

\begin{problem}
Три бруска массами $m$, $3m$ и $2m$, соответственно, толкают с силой $F$.
Бруски касаются друг друга и скользят по плоскости без трения.
Найди величину силы, с которой второй брусок действует на первый.

\begin{minipage}{0.8\textwidth}
\begin{center}
\resizebox{8.0cm}{1.6cm}{
\begin{tikzpicture}
\draw (-5,0) -- (5,0);
\draw[-{Latex[length=5mm, width=2mm]}] (-5,1) -- (-3,1) node[midway, above]{$F$};
\draw (-3,0) -- (-3,2) -- (-1,2) node[midway, below=0.8cm]{$m$} node[midway, above]{1} -- (-1,0);
\draw (-1,0) -- (-1,2) -- (1,2) node[midway, below=0.8cm]{$3m$} node[midway, above]{2} -- (1,0);
\draw (1,0) -- (1,2) -- (3,2) node[midway, below=0.8cm]{$2m$} node[midway, above]{3} -- (3,0);
\end{tikzpicture}
}
\end{center}
\end{minipage}
\end{problem}


\begin{problem}
% задачка от Ламзина
В треугольнике $ABC$ угол между биссектрисами, исходящими из углов $A$ и $B$,
равен $43^{\circ}$. Найди угол $C$ треугольника $ABC$.
\end{problem}


\begin{problem}
Три одинаковых шарика надувают гелием так, что их диаметры относятся как 1:2:3.
Известно, что первый шарик может поднять груз максимальной массой $m$,
а второй — максимальной массой $15m$. Считать плотность гелия во всех шариках одинаковой.

Какой максимальный груз может поднять третий шарик?
\end{problem}


\end{document}
