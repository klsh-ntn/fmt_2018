\documentclass[a4paper, 11pt]{article}

%\newcommand{\LastChange}{Time-stamp: "2014-07-09 15:59:43 aga SeminarProblems.tex"}



\usepackage[utf8]{inputenc}
\usepackage[russian]{babel}

% \usepackage[colorinlistoftodos]{todonotes}
\usepackage[colorlinks=true, allcolors=blue]{hyperref}

%\usepackage{euler}
%\usepackage{xltxtra} % loads: fixltx2e, metalogo, xunicode, fontspec

% \usepackage{multicol} % many columns
\usepackage{amsmath,amsfonts,amssymb,amsthm}
\usepackage{fullpage}
\usepackage{graphicx}
\usepackage{bm}
\usepackage{multicol} % текст в несколько колонок

\usepackage{marvosym} % значок мужского туалета

%\usepackage{enumerate}
\usepackage{textcomp} % text in formulas

%\usepackage{paralist}
\usepackage{enumitem} % more options for lists

\usepackage{tikz} % картинки
\usetikzlibrary{arrows.meta, quotes, angles} % tikz-прибамбас для рисовки стрелочек подлиннее

\usepackage[includehead, top=0.5cm, bottom=0.5cm, left=1cm, right=1cm]{geometry}


% \usepackage{fontspec} % что-то про шрифты?
% \usepackage{polyglossia} % русификация xelatex

% \setmainlanguage{russian}
% \setotherlanguages{english}

% download "Linux Libertine" fonts:
% http://www.linuxlibertine.org/index.php?id=91&L=1
% \setmainfont{Linux Libertine O} % or Helvetica, Arial, Cambria
% why do we need \newfontfamily:
% http://tex.stackexchange.com/questions/91507/
% \newfontfamily{\cyrillicfonttt}{Linux Libertine O}

% \defaultfontfeatures{Mapping=tex-text}

\AddEnumerateCounter{\asbuk}{\russian@alph}{щ} % для списков с русскими буквами
\setlist[enumerate, 2]{label=\asbuk*),ref=\asbuk*}

%\setmainfont{Times New Roman}
%\setmainfont{Arial}
%\setmainfont{PT Sans}


%\setlength{\topmargin}{0in}
%\setlength{\headheight}{0cm}
%\setlength{\headsep}{0in}
%\setlength{\oddsidemargin}{-0.5in}
%\setlength{\evensidemargin}{-0.5in}
%\setlength{\textwidth}{7.5in}
%\setlength{\textheight}{9.0in}


% \newcommand{\staritem}{\refstepcounter{enumi}\item[\bf *\theenumi.]}

% \newcommand{\bsym}{\boldsymbol}


%\newcommand{\FigWidth}{0.3\columnwidth}



\newtheoremstyle{break}% name
  {}%         Space above, empty = `usual value'
  {1pt}%         Space below
  {}% Body font
  {}%         Indent amount (empty = no indent, \parindent = para indent)
  {\bfseries}% Thm head font
  {.}%        Punctuation after thm head
  {\newline}% Space after thm head: \newline = linebreak
  {}%         Thm head spec

\theoremstyle{break}
\newtheorem{problem}{Задача}[subsection]
\renewcommand{\theproblem}{\arabic{problem}}% Remove subsection from the counter representation


\begin{document}

\thispagestyle{empty}
%%%%%%%%%%%%%%%%%%%%%%%%%%%%%%
%%%%%%%%%%%%%%%%%%%%%%%%%%%%%%
\subsection*{Свалка}
%%%%%%%%%%%%%%%%%%%%%%%%%%%%%%

\begin{problem}
На расстоянии $L=100$ м от стены лежит футбольный мяч.
Адо бежит к мячу с постоянной скоростью $u=10$ м/с в направлении перендикулярном плоскости стены.
Добегая до мяча, Адо наносит удар и придаёт
мячу горизонтальную скорость $v=30$ м/c относительно поверхности Земли, $v>u$.
После удара Адо продолжает движение с прежней скоростью. Столкновение мяча со стеной абсолютно упругое.

Через сколько секунд после удара Адо снова встретится с мячом?

Дорогой друг, трением мяча о поверхность Земли разрешаю тебе пренебречь! Твой главный судья $\heartsuit$.
\end{problem}

\begin{problem}
Какое наименьшее количество сомножителей нужно зачеркнуть
в произведении $1\cdot 2\cdot 3\cdot \ldots \cdot 76\cdot 77$, чтобы
оно не заканчивалось на ноль?

%
\end{problem}


\begin{problem}
В правильный треугольник со стороной $2$ вписаны две окружности равных радиусов, как показано на рисунке.

Чему равно расстояние между центрами?


\begin{minipage}{0.8\textwidth}
\begin{center}
\resizebox{3.0cm}{2cm}{
\begin{tikzpicture}
\draw (0,0) -- (2,0) -- (1,1.732) -- (0,0);
\draw (0.634,0.366) circle (0.366cm);
\draw (0.634,0.366) circle[radius=1pt];
\fill (0.634,0.366)  circle[radius=1pt];
\draw (1.366,0.366) circle[radius=1pt];
\fill (1.366,0.366)  circle[radius=1pt];
\draw (1.366,0.366) circle (0.366cm);
\end{tikzpicture}
}
\end{center}
\end{minipage}

% $2r = \sqrt{3} - 1$

\end{problem}



\begin{problem}
Пушка стреляет вертикально вверх снарядом массой $3m$.
В верхней точке траектории снаряд разрывается на два осколка, массами $m$ и $2m$.
Осколок массы $m$ сразу после взрыва вылетает горизонтально и падает на расстоянии
$L$ от пушки.

Чему равно расстояние между упавшими осколками?

% $L + L/2 = 3L/2$
\end{problem}

\subsection*{Свалка}
%%%%%%%%%%%%%%%%%%%%%%%%%%%%%%
\setcounter{problem}{0}



\begin{problem}
На расстоянии $L=100$ м от стены лежит футбольный мяч.
Адо бежит к мячу с постоянной скоростью $u=10$ м/с в направлении перендикулярном плоскости стены.
Добегая до мяча, Адо наносит удар и придаёт
мячу горизонтальную скорость $v=30$ м/c относительно поверхности Земли, $v>u$.
После удара Адо продолжает движение с прежней скоростью. Столкновение мяча со стеной абсолютно упругое.

Через сколько секунд после удара Адо снова встретится с мячом?

Дорогой друг, трением мяча о поверхность Земли разрешаю тебе пренебречь! Твой главный судья $\heartsuit$.
\end{problem}

\begin{problem}
Какое наименьшее количество сомножителей нужно зачеркнуть
в произведении $1\cdot 2\cdot 3\cdot \ldots \cdot 76\cdot 77$, чтобы
оно не заканчивалось на ноль?

%
\end{problem}


\begin{problem}
В правильный треугольник со стороной $2$ вписаны две окружности равных радиусов, как показано на рисунке.

Чему равно расстояние между центрами?


\begin{minipage}{0.8\textwidth}
\begin{center}
\resizebox{3.0cm}{2cm}{
\begin{tikzpicture}
\draw (0,0) -- (2,0) -- (1,1.732) -- (0,0);
\draw (0.634,0.366) circle (0.366cm);
\draw (0.634,0.366) circle[radius=1pt];
\fill (0.634,0.366)  circle[radius=1pt];
\draw (1.366,0.366) circle[radius=1pt];
\fill (1.366,0.366)  circle[radius=1pt];
\draw (1.366,0.366) circle (0.366cm);
\end{tikzpicture}
}
\end{center}
\end{minipage}

% $2r = \sqrt{3} - 1$

\end{problem}



\begin{problem}
Пушка стреляет вертикально вверх снарядом массой $3m$.
В верхней точке траектории снаряд разрывается на два осколка, массами $m$ и $2m$.
Осколок массы $m$ сразу после взрыва вылетает горизонтально и падает на расстоянии
$L$ от пушки.

Чему равно расстояние между упавшими осколками?

% $L + L/2 = 3L/2$
\end{problem}



\end{document}
