\documentclass[a4paper, 11pt]{article}

%\newcommand{\LastChange}{Time-stamp: "2014-07-09 15:59:43 aga SeminarProblems.tex"}


\usepackage[utf8]{inputenc}
\usepackage[russian]{babel}

% \usepackage[colorinlistoftodos]{todonotes}
\usepackage[colorlinks=true, allcolors=blue]{hyperref}

%\usepackage{euler}
%\usepackage{xltxtra} % loads: fixltx2e, metalogo, xunicode, fontspec

% \usepackage{multicol} % many columns
\usepackage{amsmath,amsfonts,amssymb,amsthm}
\usepackage{fullpage}
\usepackage{graphicx}
\usepackage{bm}
\usepackage{multicol} % текст в несколько колонок

\usepackage{marvosym} % значок мужского туалета

%\usepackage{enumerate}
\usepackage{textcomp} % text in formulas

%\usepackage{paralist}
\usepackage{enumitem} % more options for lists

\usepackage{tikz} % картинки
\usetikzlibrary{arrows.meta, quotes, angles} % tikz-прибамбас для рисовки стрелочек подлиннее

\usepackage[includehead, top=0.5cm, bottom=0.5cm, left=1cm, right=1cm]{geometry}


%\usepackage{fontspec} % что-то про шрифты?
%\usepackage{polyglossia} % русификация xelatex

%\setmainlanguage{russian}
%\setotherlanguages{english}

% download "Linux Libertine" fonts:
% http://www.linuxlibertine.org/index.php?id=91&L=1
%\setmainfont{Linux Libertine O} % or Helvetica, Arial, Cambria
% why do we need \newfontfamily:
% http://tex.stackexchange.com/questions/91507/
%\newfontfamily{\cyrillicfonttt}{Linux Libertine O}

%\defaultfontfeatures{Mapping=tex-text}

\AddEnumerateCounter{\asbuk}{\russian@alph}{щ} % для списков с русскими буквами
\setlist[enumerate, 2]{label=\asbuk*),ref=\asbuk*}

%\setmainfont{Times New Roman}
%\setmainfont{Arial}
%\setmainfont{PT Sans}


%\setlength{\topmargin}{0in}
%\setlength{\headheight}{0cm}
%\setlength{\headsep}{0in}
%\setlength{\oddsidemargin}{-0.5in}
%\setlength{\evensidemargin}{-0.5in}
%\setlength{\textwidth}{7.5in}
%\setlength{\textheight}{9.0in}


% \newcommand{\staritem}{\refstepcounter{enumi}\item[\bf *\theenumi.]}

% \newcommand{\bsym}{\boldsymbol}


%\newcommand{\FigWidth}{0.3\columnwidth}



\newtheoremstyle{break}% name
  {}%         Space above, empty = `usual value'
  {1pt}%         Space below
  {}% Body font
  {}%         Indent amount (empty = no indent, \parindent = para indent)
  {\bfseries}% Thm head font
  {.}%        Punctuation after thm head
  {\newline}% Space after thm head: \newline = linebreak
  {}%         Thm head spec

\theoremstyle{break}
\newtheorem{problem}{Задача}[subsection]
\renewcommand{\theproblem}{\arabic{problem}}% Remove subsection from the counter representation


\begin{document}

\thispagestyle{empty}
%%%%%%%%%%%%%%%%%%%%%%%%%%%%%%
%%%%%%%%%%%%%%%%%%%%%%%%%%%%%%
\subsection*{Финал}
%%%%%%%%%%%%%%%%%%%%%%%%%%%%%%

\begin{problem}
Найди положительный корень уравнения

\[
x = \sqrt{8} + \dfrac{2}{\sqrt{8} + \dfrac{2}{\sqrt{8} + \dfrac{2}{\sqrt{8} + \dfrac{2}{\sqrt{8} + \dfrac{2}{x}}}}}.
\]

% решение единственно, тк с ростом $x$ левая часть растёт, а правая — падает
% решим более простое уравнение $x = \sqrt{8} + \frac{2}{x}$
% получим два корня, $x = \sqrt{2} + 2$, корень $x=\sqrt{2} - 2$ не подходит
%
%

\end{problem}

\begin{problem}
На горизонтальном шероховатом столе лежит кусок мела массой $m$. Ему
придают скорость $v$ в горизонтальной плоскости.
При движении по столу мелок истирается, теряя массу $\eta$ на каждом метре пути.
Коэффициент трения мелка о стол равен $\mu$.

Известно, что за время движения мелок полностью истёрся.

Какой путь проехал мелок?
\end{problem}


\begin{problem}
В треугольнике $ABC$ известны стороны: $AB=c$, $BC=a$, $AC=b$.
Медианы $AA_1$, $BB_1$ и $CC_1$ пересекаются в точке $O$.
Точка $A_2$ — середина $AO$, $B_2$ — середина $BO$, $C_2$ — середниа $CO$.

Чему равна сумма квадратов сторон шестиугольника $C_1 B_2 A_1 C_2 B_1 A_2$?

% ответ $(a^2 + b^2 + c^2) / 6$

\end{problem}



\begin{problem}
Автомобиль с радиусом колеса $R$ едет со скоростью $v$. Частичка грязи прилипает
к колесу, поворачивается на угол $150^{\circ}$ и отрывается.

На каком расстоянии от своего изначального местоположения приземлится частичка?
\end{problem}

\subsection*{Финал}
%%%%%%%%%%%%%%%%%%%%%%%%%%%%%%
\setcounter{problem}{0}


\end{document}
